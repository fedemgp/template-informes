
\documentclass[a4paper,10pt]{article}

% Paquetes
\usepackage[a4paper, total={6.5in, 10in}]{geometry}
\usepackage{amsmath}
\usepackage[hidelinks]{hyperref}
\usepackage{graphicx}
\usepackage{xcolor}
\usepackage{fancyhdr}
\usepackage{enumerate}
% Idioma
\usepackage[spanish, activeacute]{babel}
\usepackage[utf8]{inputenc}
\usepackage{tikz}

%Cuando quieran agregar una imagen no tienen que poner el directorio de la
%imagen, eso se está haciendo acá, simplemente pongan el nombre de la imagen

\newcommand{\imgdir}{img}
\graphicspath{{\imgdir/}}
% No se que hace esto
\def\checkmark{\tikz\fill[scale=0.4](0,.35) -- (.25,0) -- (1,.7) -- (.25,.15) -- cycle;} 

% Configuraciones
\hypersetup{
    colorlinks,
    linkcolor={black!50!black},
    citecolor={blue!50!black},
    urlcolor={blue!80!black}
}

\begin{document}
\newcommand{\nombreMateria}{Instrumentos Electrónicos}
\newcommand{\codigoMateria}{86.13}

\newcommand{\numeroInforme}{4}
\newcommand{\nombreInforme}{Impedancímetro}

\newcommand{\nombreDos}{Gómez Peter, Federico Manuel}
\newcommand{\padronDos}{96091}
\newcommand{\mailDos}{fedemgp@gmail.com}

\newcommand{\docentes}{Zothner, Enrique\\
		Cabibbo, Daniel Horacio}
\newcommand{\fecha}{Junio 2019}


\begin{titlepage}
\centering
\begin{figure}[t]
	\centering
	{\huge  Universidad de Buenos Aires \par}
	{\huge Facultad de Ingeniería \par}
	\vspace{0.5cm}
	\includegraphics[scale=0.3]{fiuba}
    \vspace{1cm}
\end{figure}
	\vspace{1cm}
	{\huge \bfseries \nombreMateria \LARGE (\codigoMateria) \par}
	\vspace{1cm}
    {\textbf{\LARGE Informe \numeroInforme: \nombreInforme}\par}
    \vspace{2cm}
    {\underline{\Large INTEGRANTES:}\\}
    \vspace{0.5cm}
	{\nombreUno - nº \padronUno- \emph{\mailUno}\\ }
    \vspace{0.5cm}
	{\nombreDos - nº \padronDos- \emph{\mailDos}\\ }
	\vspace{0.5cm}
	{\nombreTres - nº \padronTres- \emph{\mailTres}\\ }
    \vspace{1cm}
    {\underline{\Large  DOCENTES:}\\}
    \vspace{0.5cm}
	{\Large\itshape  \docentes \par}
    \vspace{1cm}
    {\large \fecha \par}
	\vfill
\end{titlepage}

% Encabezado y pie de página
\pagestyle{fancy}
\fancyhead[C]{\rule{.5\textwidth}{4\baselineskip}}
\setlength{\headheight}{52pt}
\renewcommand{\sectionmark}[1]{\markboth{}{\thesection\ \ #1}}
\lhead{\includegraphics[width= 3 cm]{./logo_fiuba_alta.jpg}}
\chead{}
\rhead{\codigoMateria - \nombreMateria}
\lfoot{}
\cfoot{\thepage}
\rfoot{}

\newpage
\tableofcontents
\newpage

% Ver de esta lista qué no hace falta

\section{Introducción} 
\label{sec:intro}

\subsection{Desarrollo}
\label{sec:desarrollo}
\input{desarrollo.tex}

\subsection{Mediciones}
\label{sec:mediciones}
\input{mediciones.tex}

\subsection{Conclusiones}
\label{sec:conclusiones}
\input{conclusiones.tex}


\section{Bibliografía}

\end{document}
